\Chapter{Összefoglalás}

A dolgozatban bemutatott alkalmazás éppen csak az alapnak tekinthető ahhoz képest, amit el lehetne érni a játékosítással. Nyilván vannak egyszerűbb és komplexebb rendszerek. Egyszerűbbek közé sorolhatók azok a hirdetések, ahol például mozog a célkereszt és le kell lőni vele valamit. Ezzel arra motiválnak, hogy kattintsunk rá a hirdetésre. Emiatt lehet pár ember rákattint, de azért olyan nagy hatást nem gyakorol ránk, és nem fog játékfüggőséget okozni. A komplexebb alatt pedig olyan óriási méretű szofverekre gondolok, mint például a eBay \cite{ebay}, Amazon \cite{amazon}, Facebook \cite{facebook}. A legtöbb közösségi médiával kapcsolatos alkalmazásról nem is gondolnánk, hogy köze van a játékosításhoz. Pedig ez a sikerük egyik titka, nem véletlenül vagyunk képesek napi szinten órákat eltölteni velük. Rendkívül tudatosan és precízen vannak beleültetve ezek az elemek, így több felhasználói interakciót váltanak ki. Emellett lehet tényleges játékokra is gondolni, amiknek hasonlóan nagy felhasználó bázisa van. Ilyen például a Minecraft \cite{minecraft}, Clash of Clans \cite{clashOfClans} vagy Candy Crush \cite{candycrush}. Ezek is hiába jöttek már ki több éve, még mindig rengeteg játszanak vele pont amiatt, mert az előző fejezetben említett hajtóerők jól vannak implementálva. \newline

A korábbi fejezet végén azt is említettem, hogy a játékosítás nem merül ki szimpla kitűzőkben és pontokban. Viszont az itt felsorolt tervekkel együtt már egy sokkal nagyobb mértékű motiváció és elkötelezettség alakulhatna ki. \newline

Úgy gondolom, hogy akár már az eddigiek is segíthetnek egy kis löketet adni abban, hogy tanuljanak vagy csak egy fokkal élvezetesebbek legyenek a kötelező számonkérések. Egy kis fellélegzést adhat a fekete fehéren kinyomtatott dolgozat kérdések után, és motivációt nyújthat. \newline

A játékosítás azt a célt szolgálná, hogy a minket körbevevő dolgok vonzóak és szórakoztatóak legyenek számunkra. Valamint a napi életben szembejövő kihívásokra ne úgy tekintsünk, hogy ezeken minél hamarabb túl akarunk csak esni. A célja ennek az ellentétje lenne, hogy várjuk az előttünk álló kihívásokat. A játékosítás egy eszköz és egy gondolkodásmód, aminél el lehet dönteni, hogy szeretnénk használni vagy sem. Úgy gondolom, ha úgy döntünk kívánjuk alkalmazni, akkor egy élvezetesebb és összességében jobb irányba fordíthatjuk a mindennapjainkat azáltal, hogy sokkal jobban élvezzük a bizonyos feladatokkal töltött időt. Segít megerősíteni bennünk a fejlődésre törekvés vágyát és motivációt nyújtson úgy, hogy ezáltal fejlődjünk.
