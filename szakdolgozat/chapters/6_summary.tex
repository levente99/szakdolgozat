\Chapter{Összefoglalás}

% Hasonló szerepe van, mint a bevezetésnek.
% Itt már múltidőben lehet beszélni.
% A szerző saját meglátása szerint kell összegezni és értékelni a dolgozat fontosabb eredményeit.
% Meg lehet benne említeni, hogy mi az ami jobban, mi az ami kevésbé jobban sikerült a tervezettnél.
% El lehet benne mondani, hogy milyen további tervek, fejlesztési lehetőségek vannak még a témával kapcsolatban.

\Section{Jövőbeli tervek}

Az alkalmazásban egy teszt készítéséhez szükséges alap funkciók meg vannak már valósítva, mindazonáltal még rengeteg lehetőség rejlik ebben a szoftverben. Az alábbiakat lehetne még hozzáadni:

\subsubsection{Bolt}
Az alkalmazáson belül lehetne kitalálni saját pénznemet, vagy akár a pontjaiddal is fel lehetne fizetni és különböző előnyöket lehetne ebből venni. Ezek az előnyök minden kérdésnél feljönnének és lehetne őket aktiválni egyszer. Ilyen előny lehet például, hogy le lehet felezni a válaszokat, lehessen növel az időt vagy akár, hogy plusz xp-t kapjunk egy tesztre.

\subsubsection{Kitűzőket létrehozni jutalmazás miatt}
Lehetne kitűzőket gyűjteni, ebből olyanok lehetne elérni mint például tesztben 5 vagy 10 kérdésre egymás utána helyes választ adott a kitöltő, több mint 3 tesztben 500 feletti xp-t ért el vagy 3 kérdésre gyorsan adott jó választ.

\subsubsection{Tesztek összekeverése}
Az online tesztkitöltések egyik hátránya, hogy könnyebb csalni kitöltés során. Ennek megnehezítésének egyik módja az lehetne, hogy a kérdések véletlenszerű sorrendben jönnek fel mindenkinél.

\subsubsection{Avatár készítés}
A tulajdonjog és birtoklás is ugyan úgy motiválja a felhasználót, mert úgy érzik, hogy valamit birtokolnak vagy irányítanak. Amikor egy személy valamit a tulajdonának érez valamivel szemben, akkor automatikusan növelni és fejleszteni akarja azt. Ezenkívül, ha egy személy sok időt tölt profilja vagy avatárjának testre szabásával, automatikusan nagyobb felelősséget érez iránta. Emiatt bele lehetne tenni a regisztrációhoz egy avatár készítő felületet, ahol magukat vagy akár egy kitalált karaktert, hozhat létre az új felhasználó és magáénak érezheti a profilját. Ezeket az avatárok megjelennének a főoldalon és a ranglistán az eredmények mellett.

\subsubsection{Osztályok, csoportok létrehozása}
Az oldal használatához a kötődést nagyban megnövelhetné, ha közösségeket lehetne létrehozni benne. Ezalatt arra gondolok hogy a tesztkitöltők között általában van valamilyen kapcsolat, ez döntő többségben iskolához köthető kapcsolat, például egy osztályba járnak vagy egy szakon vannak az egyetemen. Emiatt hasznos lenne, ha lehetne ilyen csoportokat létrehozni ahol esetleg kérdéseket tudnának feltenni a tanárnak vagy átbeszélni a tesztel kapcsolatos dolgokat. Így motiválnák egymást a tanulásra és jobban teljesítenének. Valamint a tanároknak is nagy segítség lehetne, mivel teszt készítésnél a kitöltőknél nem egyesével kéne beírniuk az email címeket, hanem mondjuk kiválaszthatná a csoportot név alapján és hozzá rendelhetné a csoport összes tagjához.

\subsubsection{Felhasználók profiljának megtekintése}
Lehetne felhasználói profilokat is megtekinteni. Itt listázva lenne a közelmúltban szerzett eredmények, jelvényei, vagy hogy milyen csoportoknak a tagja. Ez arra szolgálna, hogy növelné a versenyszellemet a felhasználók között, mivel láthatnák, hogy ki mennyi pontot szerzett mostanában.

\Section{Összegzés}

Amiket én lefejlesztettem, ezek csak nagyon az alapjai annak, amit el lehet érni a játékosítással. Nyilván vannak egyszerűbb és komplexebb rendszerek. Egyszerűbbek közé sorolhatók azok a hirdetések, ahol például mozog a célkereszt és le kell lőni vele valamit. Ezzel arra motiválnak, hogy kattintsunk rá a hirdetésre. Emiatt lehet pár ember rákattint, de azért olyan nagy hatást nem gyakorol ránk és nem fog játékfüggőséget okozni. A komplexebb alatt pedig olyan óriási méretű szofverekre gondolok mint például a eBay \cite{ebay}, Amazon \cite{amazon}, Facebook \cite{facebook}. A legtöbb közösségi médiával kapcsolatos alkalmazásról nem is gondolnánk, hogy köze van a játékosításhoz. Pedig ez a sikerük egyik titka, nem véletlenül vagyunk képesek napi szinten órákat eltölteni vele. Rendkívül tudatosan és precízen vannak beleültetve ezek az elemek, így több felhasználói interakciót váltunk ki. Emellett lehet tényleges játékokra is gondolni, amiknek hasonlóan nagy felhasználó bázisa van. Ilyen például a Minecraft \cite{minecraft}, Clash of Clans \cite{clashOfClans} vagy Candy Crush \cite{candycrush}. Ezek is hiába jöttek már ki több éve még mindig rengeteg játszanak vele pont amiatt, mert az előző fejezetben említett hajtóerők jól le vannak implementálva. \newline

Az előző fejezet végén azt is említettem, hogy a játékosítás nem merül ki szimpla kitűzőkben és pontokban. Viszont az itt felsorolt tervekkel együtt már egy sokkal nagyobb mértékű motiváció és elkötelezettség alakulhatna ki. \newline

De én úgy gondolom, hogy akár már az eddigiek is segíthetnek egy kis löketet adni abban, hogy tanuljanak vagy csak egy fokkal élvezetesebbek legyenek ezek a kötelező számonkérések. Egy kis fellélegzést adhat a fekete fehéren kinyomtatott dolgozat kérdések után és motivációt nyújthat. \newline

A játékosítás azt a célt szolgálná, hogy a minket körbevevő dolgok vonzóak és szórakoztatóak legyenek számunkra. Valamint a napi életben szembejövő kihívásokra ne úgy tekintsünk, hogy ezeken minél hamarabb túl akarunk csak esni. A célja ennek az ellentétje lenne, hogy várjuk az előttünk álló kihívásokat. A játékosítás egy eszköz és egy gondolkodásmód, aminél el lehet dönteni, hogy szeretnénk használni vagy sem. Úgy gondolom, ha úgy döntünk kívánjuk alkalmazni, akkor egy élvezetesebb és összességében jobb irányba fordíthatjuk a mindennapjainkat, azáltal, hogy sokkal jobban élvezzük a bizonyos feladatokkal töltött időt. Segít megerősíteni bennünk a fejlődésre törekvés vágyát és motivációt nyújtson úgy, hogy ezáltal fejlődjünk.