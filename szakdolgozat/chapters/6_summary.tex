\Chapter{Összefoglalás}

% Hasonló szerepe van, mint a bevezetésnek.
% Itt már múltidőben lehet beszélni.
% A szerző saját meglátása szerint kell összegezni és értékelni a dolgozat fontosabb eredményeit.
% Meg lehet benne említeni, hogy mi az ami jobban, mi az ami kevésbé jobban sikerült a tervezettnél.
% El lehet benne mondani, hogy milyen további tervek, fejlesztési lehetőségek vannak még a témával kapcsolatban.

\Section{Jövőbeli tervek}

Az alkalmazásban egy teszt készítéséhez szükséges alap funkciók meg vannak már valósítva, mindazonáltal még rengeteg lehetőség rejlik ebben a szoftverben. Az alábbiakat lehetne még hozzáadni:

\begin{itemize}
    \item {Bolt}
          \begin{addmargin}[\parindent]{0pt}
              Az alkalmazáson belül lehetne kitalálni saját pénznemet, vagy akár a pontjaiddal is fel lehetne fizetni és különböző előnyöket lehetne ebből venni. Ezek az előnyök minden kérdésnél feljönnének és lehetne őket aktiválni egyszer. Ilyen előny lehet például hogy le lehet felezni a válaszokat, lehessen növel az időt vagy akár hogy plusz xp-t kapjunk egy tesztre.
          \end{addmargin}
    \item {Kitűzőket létrehozni jutalmazás miatt}
          \begin{addmargin}[\parindent]{0pt}
              Lehetne kitűzőket gyűjteni, ebből olyanok lehetnének hogy például egy tesztben 5 vagy 10 kérdésre egymás utána helyes választ adott a kitöltő, több mint 3 tesztben 500 feletti xp-t ért el, vagy hogy 3 kérdésre gyorsan adott jó választ.
          \end{addmargin}
    \item {Tesztek összekeverése}
          \begin{addmargin}[\parindent]{0pt}
              Az online tesztkitöltések egyik hátránya, hogy könnyebb csalni kitöltés során. Ennek megnehezítésének egyik módja az lehetne, hogy a kérdések véletlenszerű sorrendben jönnek fel mindenkinél.
          \end{addmargin}
    \item {Avatár készítés}
          \begin{addmargin}[\parindent]{0pt}
              A tulajdonjog és birtoklás is ugyan úgy motiválja a felhasználót, mert úgy érzik, hogy valamit birtokolnak vagy irányítanak. Amikor egy személy valamit a tulajdonának érez valamivel szemben, akkor automatikusan növelni és fejleszteni akarja azt. Ezenkívül, ha egy személy sok időt tölt profilja vagy avatárjának testreszabásával, automatikusan nagyobb felelősséget érez iránta. Emiatt bele lehetne tenni a regisztrációhoz egy avatár készítő felületet, ahol magukat vagy akár egy kitalált karaktert, hozhat kétre az új felhasználó és magáénak érezheti a profilját. Ezeket az avatárok megjelennének a főoldalon és a ranglistán az eredmények mellett.
          \end{addmargin}
    \item {Osztályok, csoportok létrehozása}
          \begin{addmargin}[\parindent]{0pt}
              Az oldal használatához a kötődést nagyban megnövelhetné ha közösségeket lehetne létrehozni benne. Ezalatt arra gondolok hogy a tesztkitöltők között általában van valamilyen kapcsolat, ez általában iskolához köthető kapcsolat, például egy osztályba járnak vagy egy szakon vannak az egyetemen. Emiatt hasznos lenne ha lehetne ilyen csoportokat létrehozni ahol esetleg kérdéseket tudnának feltenni a tanárnak, vagy átbeszélni a tesztel kapcsolatos dolgokat így motiválnák egymást a tanulásra és, hogy jobban teljesítsenek. Valamint a tanároknak is nagy segítség lehetne mivel teszt készítésnél a kitöltőknél nem egyesével kéne beírniuk az email címeket hanem mondjuk kiválaszthatná a csoportot név alapján és hozzá rendelhetné a csoport összes tagjához.
          \end{addmargin}
    \item {Felhasználók profiljának megtekintése}
          \begin{addmargin}[\parindent]{0pt}
              Lehetne felhasználói profilokat is megtekinteni. Itt listázva lenne a közelmúltban szerzett eredmények, jelvénjei vagy hogy milyen csoportoknak a tagja. Ez arra szolgálna hogy növelné a versenyszellemet a felhasználók között, mivel láthatnák hogy ki mennyi pontot szerzett mostanában.
          \end{addmargin}

\end{itemize}