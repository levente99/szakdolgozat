\Chapter{Bevezetés}

A \textit{gamifikáció} (vagy magyarosabb nevén a \textit{játékosítás}) lényege, a játékos elemek alkalmazása nem játékos környezetekben. Fő szerepe a felhasználói elkötelezettség, az élvezet és a hűség növelése az alkalmazás felé. Így próbál segíteni céljaink elérésében. \newline

Tehát úgy definiálhatjuk a játékosítást, hogy játékos elemekkel töltünk meg egy nem játékos elmeket tartalamazó környezetet. Ezt átfogalmazhatnánk akár úgy is, hogy az unalmasabb vagy nehezebb feladatokat élvezetesebbé próbáljuk vele tenni, így segít nekünk saját lehetőségeik kiaknázásában. Viszont ez így egy elég tág megfogalmazás. A többször említett játékelemek alatt gondolhatunk pontokra, szintekre, bármi féle jutalmakra. A helyett, hogy a felhasználó csak elvégezne egy feladatot ezek a plusz tényezők szerepet játszanak a munkafolyamatban azért, hogy motivációt nyújtsanak. Emellett ösztönöz valamilyen eredmény, online státusz elérésében, abban, hogy egy közösség része legyünk, vagy jól szerepeljünk egy megmérettetésben. \newline

Ami érdekes, hogy a gamifikáció nem új fogalom. Míg a kifejezést az utóbbi időben hozhatták létre, a koncepció az élet számos területén létezik. A játékokat és a játékhoz hasonló elemeket évezredek óta használják nevelésre, szórakozásra és elkötelezettség kialakítására. A játékosítás magába foglalja a \textit{strukturális gamifikáció}t (,,játékmechanika'') és a \textit{tartalmi gamifikáció}t (,,élménytervezés''), kiemeli az elkötelezettség és az ösztönzés építőköveit, hogy aztán ezt átültethessük más területekre. \newline

Számos gamifikációs modell jutalmazza a felhasználót a kívánt feladat elvégzéséért, majd egy ranglistát állítanak fel, így további fejlődésre ösztönözve. Ezek az elemek egy magas szintű elkötelezettséggel rendelkező gamifikációs platformot eredményeznek.
A már említett játékelemek, pontok, jelvények és ranglisták, csak a jéghegy csúcsa. \newline

A dolgozatban kifejtek pár létező megvalósítást és az azokban implementált funkciókat. Bemutatom a megtervezett saját alkalmazásom kinézetét, felépítését és az általam belerakott funkciókat. Ezután ezeket a megtervezett komponenseket részletezem és bemutatom a fejlesztés módját és menetét. A dolgozat végén kifejtem a levont következtetéseimet a témával kapcsolatban.
