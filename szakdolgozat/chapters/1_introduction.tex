\Chapter{Bevezetés}

A gamifikáció vagy magyarosabban játékosítás lényege, a játékos elemek alkalmazása nem játékos környezetekben. Fő szerepe a felhasználói elkötelezettség, az élvezet és a hűség növelése az alkalmazás felé. Így próbál segíteni céljaink elérésében. \newline

Tehát úgy definiálhatjuk a játékosítást, hogy játékos elemekkel töltünk meg egy nem játékos elmeket tartalamazó környezetet. Ezt átfogalmazhatnánk akár úgy is, hogy az unalmasabb vagy nehezebb feladatokat élvezetesebbé próbáljuk vele tenni, így segít nekünk saját lehetőségeik kiaknázásában. Viszont ez így egy elég tág megfogalmazás. Ezen a többször említett játékelemek alatt gondolhatunk pontokra, szintekre, bármi féle jutalmakra. Szóval ahelyett, hogy csak elvégzel egy feladatot ezek a plusz tényezők szerepet játszanak a munkafolyamatban azért, hogy motivációt nyújtsanak. Emellett ösztönöz, hogy elérjünk valamilyen eredményt, online státuszt, közösség része legyünk, vagy jól szerepeljünk egy megmérettetésben. \newline

Ami érdekes, hogy a gamifikáció nem új fogalom. Míg a kifejezést az utóbbi időben hozhatták létre, a koncepció az élet számos területén létezik. A játékokat és a játékhoz hasonló elemeket évezredek óta használják nevelésre, szórakozásra és elkötelezettség kialakítására. A játékosítás magába foglalja a strukturális gamifikációt ("játékmechanika") és a tartalmi gamifikációt ("élménytervezés"), kiemeli az elkötelezettség és az ösztönzés építőköveit, hogy aztán ezt átültethessük más területekre. \newline

Számos gamifikációs modell jutalmazza a felhasználót a kívánt feladat elvégzéséért, majd egy ranglistát állítanak fel, így további fejlődésre ösztönöz. Ezek az elemek egy magas szintű elkötelezettséggel rendelkező gamefikációs platformot eredményeznek.
A már említett játékelemek, pontok, jelvények és ranglisták, csak a jéghegy teteje. \newline

A dolgozatban kifejtek pár létező megvalósítást és az azokban implementált funkciókat. Bemutatom a megtervezett saját alkalmazásom kinézetét, felépítését és az általam belerakott funkciókat. Ezután ezeken a megtervezett komponenseken végig megyek és bemutatom a valós lefejlesztett verzióját, nagyobb részletességgel. Ezután kifejtem a levont következtetésem a témáról.