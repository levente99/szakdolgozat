% Itt kezdődik a dolgozat lényegi része, úgy értve, hogy a saját munka bemutatása.
% Jellemzően ebben szerepelni szoktak blokkdiagramok, a program struktúrájával foglalkozó leírások.
% Ehhez célszerű UML ábrákat (például osztály- és szekvenciadiagramokat) használni.

% Amennyiben a dolgozat inkább kutatás jellegű, úgy itt lehet konkretizálni a kutatási módszertant, a kutatás tervezett lépéseit, az indoklást, hogy mit, miért és miért pont úgy érdemes csinálni, ahogyan az a későbbiekben majd részletezésre kerül.

% Ebben a fejezetben az implementáció nem kell, hogy túl nagy szerepet kapjon.
% Ez még csak a tervezési fázis.
% (Nyilván ha olyan a téma, hogy magának az implementációnak a módjával foglalkozik, adott formális nyelvet mutat be, úgy a kódpéldákat már innen sem lehet kihagyni.)

% \Section{Táblázatok}

% Táblázatokhoz a \texttt{table} környezetet ajánlott használni.
% Erre egy minta \aref{tab:minta}. táblázat.
% A hivatkozáshoz az egyedi \texttt{label} értéke konvenció szerint \texttt{tab:} prefixszel kezdődik.

% \begin{table}[h]
% \centering
% \caption{Minta táblázat. A táblázat felirata a táblázat felett kell legyen!}
% \label{tab:minta}
% \begin{tabular}{l|c|c|}
% a & b & c \\
% \hline
% 1 & 2 & 3 \\
% 4 & 5 & 6 \\
% \hline
% \end{tabular}
% \end{table}

% \Section{Ábrák}

% Ábrákat a \texttt{figure} környezettel lehet használni.
% A használatára egy példa \aref{fig:cimer}. ábrán látható.
% Az \texttt{includegraphics} parancsba 
% Az ábrák felirata az ábra alatt kell legyen.
% Az ábrák hivatkozásához használt nevet konvenció szerint \texttt{fig:}-el célszerű kezdeni.

% \begin{figure}[h]
% \centering
% \includegraphics[scale=0.3]{images/me_logo.png}
% \caption{A Miskolci Egyetem címere.}
% \label{fig:cimer}
% \end{figure}

% \Section{További környezetek}

% A matematikai témájú dolgozatokban szükség lehet tételek és bizonyításaik megadására.
% Ehhez szintén vannak készen elérhető környezetek.

% \begin{definition}
% Ez egy definíció
% \end{definition}

% \begin{lemma}
% Ez egy lemma
% \end{lemma}

% \begin{theorem}
% Ez egy tétel
% \end{theorem}

% \begin{proof}
% Ez egy bizonyítás
% \end{proof}

% \begin{corollary}
% Ez egy tétel
% \end{corollary}

% \begin{remark}
% Ez egy megjegyzés
% \end{remark}

% \begin{example}
% Ez egy példa
% \end{example}

\Chapter{Tervezés}

% TODO: Érdemes lenne részletezni magát az adatmodellt is, tehát hogy egy felhasználóhoz, feladathoz mik tartoznak. (Lehet a felületi tervek előtt vagy után is, szokták így-is úgy-is.)



\Section{Kinézet}
A kinézetet játékos, egyszerű és átláthatóra szeretném csinálni.
Mindenhol játékos ikonokat és színvilágot szeretnék használni, az oldal ikonja az alábbi lenne:

\begin{figure}[h]
    \centering
    \includegraphics[height=5cm]{images/gameboy.png}
\end{figure}


Minden grafikus elem szeretném hogy egységes legyen így törekedtem arra hogy hasonló stílusú legyen minden. Az egyéb funkciókhoz vagy oldalakhoz társított ikonok, amelyek az oldalt még játékosabbá tennék pedig az alábbiak lennének:

\begin{figure}[h]
    \begin{subfigure}{.1\textwidth}
        \centering
        \includegraphics[width=.8\linewidth]{images/icons/003-swords.png}
    \end{subfigure}
    \begin{subfigure}{.1\textwidth}
        \centering
        \includegraphics[width=.8\linewidth]{images/icons/005-fists.png}
    \end{subfigure}
    \begin{subfigure}{.1\textwidth}
        \centering
        \includegraphics[width=.8\linewidth]{images/icons/006-gamer.png}
    \end{subfigure}
    \begin{subfigure}{.1\textwidth}
        \centering
        \includegraphics[width=.8\linewidth]{images/icons/007-gamer.png}
    \end{subfigure}
    \begin{subfigure}{.1\textwidth}
        \centering
        \includegraphics[width=.8\linewidth]{images/icons/014-computer.png}
    \end{subfigure}
    \begin{subfigure}{.1\textwidth}
        \centering
        \includegraphics[width=.8\linewidth]{images/icons/028-trophy.png}
    \end{subfigure}
    \begin{subfigure}{.1\textwidth}
        \centering
        \includegraphics[width=.8\linewidth]{images/icons/030-flag.png}
    \end{subfigure}
    \begin{subfigure}{.1\textwidth}
        \centering
        \includegraphics[width=.8\linewidth]{images/icons/052-rank.png}
    \end{subfigure}
    \begin{subfigure}{.1\textwidth}
        \centering
        \includegraphics[width=.8\linewidth]{images/icons/053-sword.png}
    \end{subfigure}
    \begin{subfigure}{.1\textwidth}
        \centering
        \includegraphics[width=.8\linewidth]{images/icons/054-sword.png}
    \end{subfigure}
    \begin{subfigure}{.1\textwidth}
        \centering
        \includegraphics[width=.8\linewidth]{images/icons/057-computer.png}
    \end{subfigure}
    \begin{subfigure}{.1\textwidth}
        \centering
        \includegraphics[width=.8\linewidth]{images/icons/058-potion.png}
    \end{subfigure}
\end{figure}

Az ikonokat a Flaticon nevű oldalról töltöttem le \cite{competitiveGamingIcon}.
% TODO: Leírni, hogy az ikonok honnan származnak.

Valamint Bootstrap-et szeretnék használni ami egy olyan keretrendszer, amely segít a weboldalak gyorsabb és könnyebb megtervezésében. HTML és CSS alapú tervezősablonokat tartalmaz a tipográfiához, űrlapokat, gombokat, táblázatokat, navigációt, modelleket stb. Ez segítene abban hogy az oldalon egységes kinézetet hozhassak létre valamit a Bootstrap CSS-je alkalmazkodik a telefonokhoz, táblagépekhez és asztali számítógépekhez is.

Ehhez egy ingyenes Bootstrap témát választottam ami szerintem menne a weboldal témájához, ezt a témát Litera-nak\cite{litera} hívják.

\Section{Az oldal felépítése}

A funkciók elrendezésének és felépítésének bemutatására képernyőterv vázlatot magyarul drótváznak, angolul wireframe vagy mockup-nak is nevezzük.
Először azt az oldalt mutatom be amivel mindenki először találkozik az oldal megnyitásakor, ez pedig a bejelentkezési és regisztrációs felület.

\begin{figure}[h!]
    \centering
    \includegraphics[width=\linewidth]{images/login_wireframe.png}
\end{figure}

\begin{figure}[h!]
    \centering
    \includegraphics[width=\linewidth]{images/signin_wireframe.png}
\end{figure}

Bejelentkezni email cím és jelszóval lehet. Regisztrációkor pedig ki kell választani hogy tanár vagy diákként szeretnénk regisztrálni.
Ezután ha sikeresen beléptünk vagy regisztráltunk akkor hozzáférhetünk az oldalhoz ami így fog kinézni:

\newpage

\begin{figure}[h!]
    \centering
    \includegraphics[width=\linewidth]{images/main_login_wireframe.png}
\end{figure}

Itt láthatjuk a felhasználónevünket, hogy mennyi xp-vel rendelkezünk, hanyas szintűek vagyunk, és hogy hány darab kitöltött és kitöltetlen tesztünk van még.


\begin{figure}[h!]
    \centering
    \includegraphics[width=\linewidth]{images/make_test_wireframe.png}
\end{figure}

Ezen a képen a tesztkészítési felület látható amelyhez majd csak a tanárok férhetnek hozzá. Egy tesztnél meghatározható hogy mi legyen a kérdés, mennyi idő legyen rá, mennyi jutalmat lehessen érte kapni, vagyis xp-t és hogy mik legyenek a válaszok és azok közül melyik a jó.

\newpage

\begin{figure}[h!]
    \centering
    \includegraphics[width=\linewidth]{images/make_test2_wireframe.png}
\end{figure}

Ezután ha elkészítettünk egy kérdést az "Új kérdés" gomb megnyomásával lehet majd új kérdést hozzáadni, itt választhatunk hogy igaz/hamis vagy kvíz típusú kérdést szeretnénk feltenni.

\begin{figure}[h!]
    \centering
    \includegraphics[width=\linewidth]{images/make_test3_wireframe.png}
\end{figure}

Az igaz/hamis típusú kérdésnél csak annyi változik hogy nem lehet válasz lehetőséget írni csak azt hogy az állítás igaz vagy hamis.

\begin{figure}[h!]
    \centering
    \includegraphics[width=\linewidth]{images/make_test4_wireframe.png}
\end{figure}

\newpage

Az utolsó lépés pedig az hogy adunk címet a tesztnek, ilyen névvel fog megjelenni majd a diákoknak. Ezután hozzáadunk tetszőleges számú diákot akiktől szeretnénk hogy töltsék ki majd, egy kitöltési határidőt, és elkészült a tesztünk.

\Section{Adatmodell}

Az adatokat egyed-kapcsolat diagrammon ábrázoltam így a teljes kép könnyebben áttekinthető ezen rendszervázlat alapján.

\begin{figure}[h!]
    \centering
    \includegraphics[width=\linewidth]{images/TestME_ER.png}
\end{figure}

A diagramm 5 táblából áll, melyek az alábbiak:
\begin{itemize}
    \item {User}
          \begin{addmargin}[\parindent]{0pt}
              Ez a tábla egy felhasználót reprezentál akinek az alap létfontosságú adatait bekérjük regisztrációkor, ilyen például a név, email cím, jelszó. Majd ezen felül még meg kell határoznia milyen szerepkörben szeretné használni az oldalt és miután töltött ki tesztet növekedni fog az xp-je mennyisége így van egy Xp adattag is.
          \end{addmargin}

    \item {Test}
          \begin{addmargin}[\parindent]{0pt}
              A Test tábla egy tanár által készített tesztet reprezentál. Egy teszt kétféle típusú lehet az egyik a kvíz a másik az igaz/hamis, így ezen kettő idegen kulcsa lesz benne. Ezenkívül a teszt címe, hogy mikor készült és hogy mi a kitöltési határidő.
          \end{addmargin}

    \item {FillingOut}
          \begin{addmargin}[\parindent]{0pt}
              Ez egy kapcsolótáblát képez a Test és a User tábla között. A kapcsolótábla azonosítója a két idegen kulcsból képzett összetett kulcs lesz. Tehát ebbe benne lesz az előbb említett két tábla idegen kulcsa, a konkrét teszt töltése után kapott Xp és az hogy mikor lett teljesítve a teszt.
          \end{addmargin}

    \item {QuizQuestion}
          \begin{addmargin}[\parindent]{0pt}
              A QuizQuestion egy kvíz kérdés adatait tárolja amibe beletartozik a kérdés, a négy válasz az idő és hogy melyik a helyes válasz.
          \end{addmargin}

    \item {TrueFalseQuestion}
          \begin{addmargin}[\parindent]{0pt}
              A TrueFalseQuestion pedig egy igaz/hamis típusú kérdés kérdését, válaszát, idejét, Xp-jét és helyes válaszát tárolja majd.
          \end{addmargin}
\end{itemize}

\Section{API}

Az alábbi endpointokat szeretném használni az adatok lekérésére és tárolására:
\begin{itemize}
    \item {Get}
          \begin{addmargin}[\parindent]{0pt}
              \url{api/user/id} \\
              \url{api/tests/id}
          \end{addmargin}

    \item {Post}
          \begin{addmargin}[\parindent]{0pt}
              \url{api/user} \\
              \url{api/fillingOut/id} \\
              \url{api/tests/id}
          \end{addmargin}
\end{itemize}

\Section{Funkciók}

\begin{itemize}
    \item {Bejelentkezés/Regisztráció}
          \begin{addmargin}[\parindent]{0pt}
              Az weboldal megnyitásakor a felhasználónak be kell jelentkeznie vagy egy helyes e-mail címmel és egy kellően biztonságos jelszóval, regisztrálni kell ha használni szeretné az oldalt.
          \end{addmargin}
    \item {Tesztkészítés}
          \begin{addmargin}[\parindent]{0pt}
              Kahoot!-hoz hasonló teszt készítési felületet szeretnék létrehozni ahol az elkészített teszteket hozzá lehet rendelni diákokhoz és ezután kitölthetik azokat.
          \end{addmargin}
    \item {Pontgyűjtés}
          \begin{addmargin}[\parindent]{0pt}
              Minden regisztrált felhasználó rendelkezik majd egy szinttel és egy bizonyos pontszámmal, amelyet a tesztek kitöltésével szerezhetnek.
          \end{addmargin}
    \item {Ranglista}
          \begin{addmargin}[\parindent]{0pt}
              Ranglista a teszt teljes kitöltését követően alakul ki a legtöbbet szerzett pontok alapján.
          \end{addmargin}
    \item {Felhasználói szerepkörök kezelése}
          \begin{addmargin}[\parindent]{0pt}
              Különböző szerepkörök lennének amik azzal bírnának hogy egy tanár jogosultságú felhasználó hozhat létre tesztet és hozzá rendelheti diákokhoz. A diák pedig csak kitölthetné azt.
          \end{addmargin}
    \item {Tesztek diákhoz rendelése}
          \begin{addmargin}[\parindent]{0pt}
              Teszt létrehozása során email cím vagy valamilyen más egyedi azonosító segítségével hozzá lehetne rendelni diákokhoz a tesztet és így értesülnének róla hogy ki kell tölteniük.
          \end{addmargin}
    \item {Összes hozzárendelt teszt megtekintése}
          \begin{addmargin}[\parindent]{0pt}
              Diákként meg lehet nézni az összes olyan tesztet amit valaki hozzá rendelt a profilhoz. És ezzel együtt a régiek eredményét is, hogy az hogy sikerült mindenkinek.
          \end{addmargin}
\end{itemize}

% TODO: Külön szakaszban érdemes lenne a szerepköröket is részletezni, az azokhoz tartozó jogosultságokat. (Lehet inkább ehhez tartozna inkább a use case diagram.)

\subsection{Szerepkörök és használati eset}

Itt látható egy kezdetleges használati eset-modell (use case diagram) ebben benne van a felhasználók információs igényeinek elemzése, funkcionális követelmények elemzése és a modell tartalmazza a rendszerrel szemben támasztott felhasználói követelményeket.
Látszik hogy kik (esetünkben tanár és diák) és mire akarják használni a rendszert.

\begin{figure}[h]
    \centering
    \includegraphics[width=12cm]{images/use_case.png}
    \caption{Használati eset-modell}
\end{figure}

\vspace{5mm}

Az oldalon lévő minden funkciót csak regisztráció vagy bejelentkezés után lehet használni. Ez azért fontos hiszen csak így lehet a felhasználónak tesztet küldeni és kitöltés után csak így lehet hozzáadni a profiljához a pontokat és megjeleníteni a nevét a ranglistán.

\vspace{5mm}

\noindent Regisztrálni lehet majd tanár és diákként. Ez a kettő azért van szétválasztva hogy a diákok jogosultsági körét korlátozni lehessen, például hogy egy diák ne írhasson egy tesztet és rendelje hozzá a tanárjához. Tehát ez a két funkció csak tanárok számára elérhető.

\subsubsection{Tanár szerepkör}
Egy tanár tud teszteket létrehozni és azokat hozzárendelni diákokhoz.

\subsubsection{Diák szerepkör}
Annyiban tér el a tanárétól hogy nem tud tesztet létrehozni csak a hozzárendelteket megtekinteni és kitölteni.
